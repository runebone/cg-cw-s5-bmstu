\section{Исследовательская часть}

% В данном разделе будут приведены примеры работы программ и сравнительный анализ алгоритмов на основе полученных данных.

\subsection{Технические характеристики}

Технические характеристики устройства, на котором выполнялось тестирование:
\begin{itemize}
    \item Процессор: AMD Ryzen 7 4700U 2.0 ГГц \cite{amd};
    \item Видеокарта: Radeon Graphics (встроенная);
    \item Оперативная память: 8 ГБ, DDR4, 3200 МГц;
    \item Операционная система: NixOS 23.11.1209.993d7 \cite{nixos};
    \item Версия ядра: 6.1.64.
\end{itemize}

Тестирование проводилось на компьютере, включенном в сеть электропитания.
Во время тестирования ноутбук был нагружен только встроенными приложениями окружения, а также непосредственно самим тестируемым приложением.

\subsection{Средства проведения опытов}

В листинге \ref{lst:get_time} приведена функция, используемая для получения процессорного времени в рамках проведения опытов. В листинге \ref{lst:bench_time} приведён пример её использования в рамках проведения опытов.

\begin{code}
    \begin{lstinputlisting}[
            label={lst:get_time},
            caption={Функция получения процессорного времени},
        ]{lst/get_time.cpp}
    \end{lstinputlisting}
\end{code}

\begin{code}
    \begin{lstinputlisting}[
            label={lst:bench_time},
            caption={Проведение замеров времени},
        ]{lst/bench_time.cpp}
    \end{lstinputlisting}
\end{code}

gRenderTime --- глобальная переменная, из которой по запросу проводилось считывание времени генерации кадра в процессе проведения опытов.

\subsection{Описание проводимых опытов}

Далее следует описание опытов, проведённых в рамках исследования.

\subsubsection{Опыт 1}

Цель опыта --- получить <<базовое>> время генерации кадра, с которым впоследствии будет сравниваться времени генерации кадра из других опытов.

Характеристики опыта:
\begin{itemize}
    \item Количество объектов на сцене: $1 + 5 N, N = \overline{0; 25}$
    \item Типы объектов сцены: $1 + 5 N$ кубов
    \item Общее количество треугольников объектов сцены: $12 \cdot (1 + 5 N)$
    \item Столкновения: Нет
    \item Количество вызовов OpenGL-функций отрисовки: $1 + 5 N$
\end{itemize}

\subsubsection{Опыт 2}

Цель опыта --- определение влияния количества вызовов OpenGL-функций отрисовки на время генерации кадра, в случае, когда сцена состоит из одних кубов.

Предположение --- чем больше вызовов OpenGL-функций, тем дольше будет генерироваться кадр.

Характеристики опыта:
\begin{itemize}
    \item Количество объектов на сцене: $1 + 5 N, N = \overline{0; 25}$
    \item Типы объектов сцены: $1 + 5 N$ кубов
    \item Общее количество треугольников объектов сцены: $12 \cdot (1 + 5 N)$
    \item Столкновения: Нет
    \item Количество вызовов OpenGL-функций отрисовки: $12 \cdot (1 + 5 N)$
\end{itemize}

\subsubsection{Опыт 3}

Цель опыта --- определение влияния количества столкновений на время рендеринга.

Предположение --- количество столкновений не влияет на время рендеринга, потому что время выполнения функции $engine.update()$ в рамках проведения опыта не учитывается.

Характеристики опыта:
\begin{itemize}
    \item Количество объектов на сцене: $1 + 5 N, N = \overline{0; 25}$
    \item Типы объектов сцены: $1 + 5 N$ кубов
    \item Общее количество треугольников объектов сцены: $12 \cdot (1 + 5 N)$
    \item Столкновения: Да
    \item Количество вызовов OpenGL-функций отрисовки: $1 + 5 N$
\end{itemize}

\subsubsection{Опыт 4}

Цель опыта --- определение влияния количества треугольников моделей объектов сцены на время генерации кадра.

Предположение --- чем больше треугольников в моделях объектов, тем дольше будет генерироваться кадр.

Характеристики опыта:
\begin{itemize}
    \item Количество объектов на сцене: $1 + 5 N, N = \overline{0; 25}$
    \item Типы объектов сцены: $1$ куб, $5 N$ чайников
    \item Общее количество треугольников объектов сцены: $12 + 6320 \cdot 5 N$
    \item Столкновения: Нет
    \item Количество вызовов OpenGL-функций отрисовки: $1 + 5 N$
\end{itemize}

\subsubsection{Опыт 5}

Цель опыта --- определение влияния количества вызовов OpenGL-функций отрисовки на время генерации кадра, в случае, когда сцена состоит преимущественно из моделей с большим количеством треугольников.

Предположение --- чем больше вызовов OpenGL-функций, тем дольше будет генерироваться кадр.

Характеристики опыта:
\begin{itemize}
    \item Количество объектов на сцене: $1 + 5 N, N = \overline{0; 25}$
    \item Типы объектов сцены: $1$ куб, $5 N$ чайников
    \item Общее количество треугольников объектов сцены: $12 + 6320 \cdot 5 N$
    \item Столкновения: Нет
    \item Количество вызовов OpenGL-функций отрисовки: $12 + 6320 \cdot 5 N$
\end{itemize}

\subsubsection{Опыт 6}

Цель опыта --- определение влияния количества столкновений на время рендеринга, в случае, когда сцена состоит преимущественно из моделей с большим количеством треугольников.

Предположение --- количество столкновений не влияет на время рендеринга, потому что время выполнения функции $engine.update()$ в рамках проведения опыта не учитывается.

Характеристики опыта:
\begin{itemize}
    \item Количество объектов на сцене: $1 + 5 N, N = \overline{0; 25}$
    \item Типы объектов сцены: $1$ куб, $5 N$ чайников
    \item Общее количество треугольников объектов сцены: $12 + 6320 \cdot 5 N$
    \item Столкновения: Да
    \item Количество вызовов OpenGL-функций отрисовки: $1 + 5 N$
\end{itemize}

\subsection{Проведение опытов}

На рисунках \ref{fig:cst} -- \ref{fig:tc} представлены снимки экрана, сделанные во время проведения опытов.
В таблицах \ref{tab:test1} -- \ref{tab:test6} представлены количественные данные, полученные в результате проведения опытов.
На рисунках \ref{fig:1234:tr} -- \ref{fig:456:time} представлены графики, построенные по результатам проведённых опытов.

\newpage

\begin{figure}[H]
	\centering
    \includegraphics[width=\textwidth]{img/cube-static-triangles.png}
	\caption{Проведение опытов 1, 2}
	\label{fig:cst}
\end{figure}

\begin{figure}[H]
	\centering
    \includegraphics[width=\textwidth]{img/cube-collision.png}
	\caption{Проведение опыта 3}
	\label{fig:cc}
\end{figure}

\begin{figure}[H]
	\centering
    \includegraphics[width=\textwidth]{img/teapot-static.png}
	\caption{Проведение опытов 4, 5}
	\label{fig:ts}
\end{figure}

\begin{figure}[H]
	\centering
    \includegraphics[width=\textwidth]{img/teapot-collision.png}
	\caption{Проведение опыта 6}
	\label{fig:tc}
\end{figure}

\begin{table}[H]
    \caption{Количественные данные, полученные в результате проведения опыта 1}
    \label{tab:test1}
    \begin{adjustbox}{width=1\textwidth}
        % \begin{tabular}{rrrrr}
\newcolumntype{P}[1]{>{\centering\arraybackslash}p{#1}}
\begin{tabular}{|P{.20\textwidth}|P{.20\textwidth}|P{.20\textwidth}|P{.20\textwidth}|P{.20\textwidth}|}
\hline
 Количество объектов & Количество треугольников & Количество коллизий & Количество вызовов функций отрисовки OpenGL & Время генерации кадра, мс\\
\hline
        26 &   312 &   0 & 26 & 1.62 \\ \hline
        46 &   552 &   0 & 46 & 2.67 \\ \hline
        86 &  1032 &   0 & 86 & 4.73 \\ \hline
       106 &  1272 &   0 & 106 & 5.75 \\ \hline
       126 &  1512 &   0 & 126 & 7.12 \\ \hline
\end{tabular}

    \end{adjustbox}
\end{table}

\begin{table}[H]
    \caption{Количественные данные, полученные в результате проведения опыта 2}
    \label{tab:test2}
    \begin{adjustbox}{width=1\textwidth}
        % \begin{tabular}{rrrrr}
\newcolumntype{P}[1]{>{\centering\arraybackslash}p{#1}}
\begin{tabular}{|P{.20\textwidth}|P{.20\textwidth}|P{.20\textwidth}|P{.20\textwidth}|P{.20\textwidth}|}
\hline
 Количество объектов & Количество треугольников & Количество коллизий & Количество вызовов функций отрисовки OpenGL & Время генерации кадра, мс\\
\hline
        26 &   312 &   0 & 312 & 1.70 \\ \hline
        46 &   552 &   0 & 552 & 2.64 \\ \hline
        86 &  1032 &   0 & 1032 & 4.77 \\ \hline
       106 &  1272 &   0 & 1272 & 5.87 \\ \hline
       126 &  1512 &   0 & 1512 & 7.20 \\ \hline
\end{tabular}

    \end{adjustbox}
\end{table}

\begin{table}[H]
    \caption{Количественные данные, полученные в результате проведения опыта 3}
    \label{tab:test3}
    \begin{adjustbox}{width=1\textwidth}
        % \begin{tabular}{rrrrr}
\newcolumntype{P}[1]{>{\centering\arraybackslash}p{#1}}
\begin{tabular}{|P{.20\textwidth}|P{.20\textwidth}|P{.20\textwidth}|P{.20\textwidth}|P{.20\textwidth}|}
\hline
 Количество объектов & Количество треугольников & Количество коллизий & Количество вызовов функций отрисовки OpenGL & Время генерации кадра, мс\\
\hline
        26 &   312 & 36.32 & 26 & 1.71 \\ \hline
        46 &   552 & 63.25 & 46 & 2.75 \\ \hline
        86 &  1032 & 117.39 & 86 & 4.61 \\ \hline
       106 &  1272 & 136.74 & 106 & 5.57 \\ \hline
       126 &  1512 & 176.61 & 126 & 6.99 \\ \hline
\end{tabular}

    \end{adjustbox}
\end{table}

% \begin{table}[H]
%     \caption{Количественные данные, полученные в результате проведения теста 3*}
%     \label{tab:test3x}
%     \begin{adjustbox}{width=1\textwidth}
%         % \begin{tabular}{rrrrr}
\newcolumntype{P}[1]{>{\centering\arraybackslash}p{#1}}
\begin{tabular}{|P{.20\textwidth}|P{.20\textwidth}|P{.20\textwidth}|P{.20\textwidth}|P{.20\textwidth}|}
\hline
 Количество объектов & Количество треугольников & Количество коллизий & Количество вызовов функций отрисовки OpenGL & Время генерации кадра, мс\\
\hline
        26 &   312 &  24.95 &  26 &  2.35 \\ \hline
        46 &   552 &  43.00 &  46 &  4.72 \\ \hline
        86 &  1032 & 113.60 &  86 & 11.72 \\ \hline
       106 &  1272 & 149.59 & 106 & 16.17 \\ \hline
       126 &  1512 & 190.76 & 126 & 21.83 \\ \hline
\end{tabular}

%     \end{adjustbox}
% \end{table}

\begin{table}[H]
    \caption{Количественные данные, полученные в результате проведения опыта 4}
    \label{tab:test4}
    \begin{adjustbox}{width=1\textwidth}
        \begin{tabular}{rrrrr}
\toprule
 n\_objects &  n\_triangles &  n\_collisions &  n\_draw\_calls &      time\_ns \\
\midrule
        26 &     158012.0 &           0.0 &          26.0 & 1.853701e+06 \\
        46 &     284412.0 &           0.0 &          46.0 & 2.903651e+06 \\
        86 &     537212.0 &           0.0 &          86.0 & 5.006796e+06 \\
       106 &     663612.0 &           0.0 &         106.0 & 5.983362e+06 \\
       126 &     790012.0 &           0.0 &         126.0 & 7.388525e+06 \\
\bottomrule
\end{tabular}

    \end{adjustbox}
\end{table}

\begin{table}[H]
    \caption{Количественные данные, полученные в результате проведения опыта 5}
    \label{tab:test5}
    \begin{adjustbox}{width=1\textwidth}
        \begin{tabular}{rrrrr}
\toprule
 Количество объектов & Количество треугольников & Количество коллизий & Количество вызовов функций отрисовки OpenGL | update & Время генерации кадра (нс)\\
\midrule
        26 & 158012.0 &   0.0 & 157259.619048 & 4.572557e+07 \\
        46 & 284412.0 &   0.0 & 283322.344828 & 7.297455e+07 \\
        86 & 537212.0 &   0.0 & 535548.842105 & 1.257912e+08 \\
       106 & 663612.0 &   0.0 & 661354.857143 & 1.599587e+08 \\
       126 & 790012.0 &   0.0 & 787139.272727 & 1.902826e+08 \\
\bottomrule
\end{tabular}

    \end{adjustbox}
\end{table}

\begin{table}[H]
    \caption{Количественные данные, полученные в результате проведения опыта 6}
    \label{tab:test6}
    \begin{adjustbox}{width=1\textwidth}
        \begin{tabular}{rrrrr}
\toprule
 n\_objects &  n\_triangles &  n\_collisions &  n\_draw\_calls &      time\_ns \\
\midrule
        26 &     158012.0 &     25.000000 &          26.0 & 1.775838e+06 \\
        46 &     284412.0 &     45.000000 &          46.0 & 2.885995e+06 \\
        86 &     537212.0 &    122.500000 &          86.0 & 4.909696e+06 \\
       106 &     663612.0 &    207.222222 &         106.0 & 5.879119e+06 \\
       126 &     790012.0 &    355.000000 &         126.0 & 7.437404e+06 \\
\bottomrule
\end{tabular}

    \end{adjustbox}
\end{table}

\newpage

\begin{figure}[H]
	\centering
    \includegraphics[width=0.9\textwidth]{img/1234/plot_triangles.pdf}
	\caption{Зависимость количества треугольников от количества объектов в опытах 1 -- 4}
	\label{fig:1234:tr}
\end{figure}

\begin{figure}[H]
	\centering
    \includegraphics[width=0.9\textwidth]{img/1234/plot_collisions.pdf}
	\caption{Зависимость количества коллизий от количества объектов в опытах 1 -- 4}
	\label{fig:1234:col}
\end{figure}

\begin{figure}[H]
	\centering
    \includegraphics[width=0.9\textwidth]{img/1234/plot_draw_calls.pdf}
	\caption{Зависимость количества вызовов OpenGL-функций отрисовки от количества объектов в опытах 1 -- 4}
	\label{fig:1234:dc}
\end{figure}

\begin{figure}[H]
	\centering
    \includegraphics[width=0.9\textwidth]{img/1234/plot_time.pdf}
	\caption{Зависимость времени генерации кадра от количесва объектов в опытах 1 -- 4}
	\label{fig:1234:time}
\end{figure}

\begin{figure}[H]
	\centering
    \includegraphics[width=0.9\textwidth]{img/456/plot_triangles.pdf}
	\caption{Зависимость количества треугольников от количества объектов в опытах 4 -- 6}
	\label{fig:456:tr}
\end{figure}

\begin{figure}[H]
	\centering
    \includegraphics[width=0.9\textwidth]{img/456/plot_collisions.pdf}
	\caption{Зависимость количества коллизий от количества объектов в опытах 4 -- 6}
	\label{fig:456:col}
\end{figure}

\begin{figure}[H]
	\centering
    \includegraphics[width=0.9\textwidth]{img/456/plot_draw_calls.pdf}
	\caption{Зависимость количества вызовов OpenGL-функций отрисовки от количества объектов в опытах 4 -- 6}
	\label{fig:456:dc}
\end{figure}

\begin{figure}[H]
	\centering
    \includegraphics[width=0.9\textwidth]{img/456/plot_time.pdf}
	\caption{Зависимость времени генерации кадра от количесва объектов в опытах 4 -- 6}
	\label{fig:456:time}
\end{figure}

% \begin{figure}[H]
% 	\centering
%     \includegraphics[width=0.9\textwidth]{img/3x/plot_collisions.pdf}
% 	\caption{Зависимость количества коллизий от количества объектов в тестах 1, 2, 3*, 4}
% 	\label{fig:1234:colx}
% \end{figure}

% \begin{figure}[H]
% 	\centering
%     \includegraphics[width=0.9\textwidth]{img/3x/plot_time.pdf}
% 	\caption{Зависимость времени генерации кадра от количесва объектов в тестах 1, 2, 3*, 4}
% 	\label{fig:1234:timex}
% \end{figure}

\subsection*{Вывод}

% Количественная оценка результатов
% Оказалось, что А во столько-то раз эффективнее Б на таких-то данных
