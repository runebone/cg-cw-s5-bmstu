\section{Аналитическая часть}

% После прочтения аналит. части человек должен сделать вывод о применимости конкретного алгоритма в его конкретной ситуации

% В аналитической части описать условия применимости алгоритма в принципе (примеры и контрпримеры - при наличии доступного объема текста)

% Таблица в конце аналитического раздела

% Констр. + аналит.: на вход - информация о полигонах

\subsection{Описание объектов сцены}

\subsection{Выбор способа задания объектов сцены}

\subsubsection{Каркасная модель}

\subsubsection{Поверхностная модель}

\subsubsection{Твердотельная модель}

\subsubsection*{Вывод}

\subsection{Выбор алгоритма удаления невидимых линий и поверхностей}

\subsubsection{Алгоритм Робертса}

\subsubsection{Алгоритм обратной трассировки лучей}

\subsubsection{Алгоритм Варнока}

\subsubsection{Алгоритм, использующий z-буфер}

\subsubsection*{Вывод}

\subsection{Выбор алгоритмов обнаружения коллизий}

\subsubsection{Алгоритм обнаружения коллизий с использованием сфер}

\subsubsection{Алгоритм AABB}

\subsubsection{Алгоритм OBB}

\subsubsection{Алгоритм GJK}

\subsubsection*{Вывод}

\subsection{Выбор алгоритмов разрешения коллизий}

\subsubsection{Алгоритм EPA}

\subsubsection*{Вывод}

\subsection{Выбор модели освещения}

\subsubsection*{Вывод}
