\section{Аналитическая часть}

% После прочтения аналит. части человек должен сделать вывод о применимости конкретного алгоритма в его конкретной ситуации

% В аналитической части описать условия применимости алгоритма в принципе (примеры и контрпримеры - при наличии доступного объема текста)

% Таблица в конце аналитического раздела

% Констр. + аналит.: на вход - информация о полигонах

В данной части под <<сценой>> понимается виртуальное пространство, в котором расположены объекты, предназначенные для визуализации.

\subsection{Описание объектов сцены}

В данном подразделе будут описаны свойства объектов, которыми он должен обладать для моделирования его движения и столкновения с другими объектами.

Для визуализации объекта, он должен содержать следующую информацию:
\begin{itemize}
    \item геометрическая информация, какую форму имеет объект;
    \item информация о местоположении в пространстве (с учётом поворота и масштабирования);
    \item информация о цвете и/или текстуре объекта.
\end{itemize}

Для моделирования движения и столкновения объектов, объекты должны содержать следующую информацию:
\begin{itemize}
    \item информация о <<коллайдере>> (как правило, упрощённая форма исходного объекта, которая используется при обнаружении столкновений);
    \item информация о физических свойствах объекта (скорость, ускорение, масса).
\end{itemize}

\subsection{Выбор представления объектов сцены}

Далее будут рассмотрены возможные способы представления объектов сцены.

\subsubsection{Каркасная модель}

Объект представляется с помощью его вершин и рёбер, без каких-либо поверхностей.

Преимущества:
\begin{itemize}
    \item простота;
    \item позволяет получить базовое представление о форме объекта;
    \item быстрая визуализация.
\end{itemize}

Недостатки:
\begin{itemize}
    \item не позволяет производить реалистическое освещение, для которого требуется информация о гранях объекта;
    \item не содержит информации, нужной для обнаружения столкновений.
\end{itemize}

\subsubsection{Поверхностная модель}

Объект представляется аппроксимированно, в виде набора поверхностей.
% Поверхность можно задать разными способами, например, уравнением или системой уравнений.
Набор поверхностей можно задать как аналитически (уравнением или системой уравнений), так и в виде полигональной сетки.
% Часто поверхность представляется в виде набора граней (обычно треугольных), которые, в свою очередь, состоят из набора вершин исходного объекта.
Часто бывает проще задать поверхности в виде полигональной сетки: набора граней и набора вершин, из которых грани состоят.

Преимущества:
\begin{itemize}
    \item простота;
    \item возможность учёта освещения;
    \item возможность достижения высокого уровня реализма;
    \item содержит информацию о поверхностях, которая нужна при обнаружении столкновений объектов;
\end{itemize}

Недостатки:
\begin{itemize}
    \item не математически точное представление, аппроксимация.
\end{itemize}

\subsubsection{Твердотельная модель}

Существует несколько методов представления твердотельных моделей: метод констркутивного представления (англ. Constructive representation, сокращённо C -- rep) и метод граничного представления (англ. Boundary representation, сокращённо B -- rep).
Оба метода предоставляют наиболее полное описание объекта, включая его внешнюю форму и внутреннюю структуру.

Преимущества:
\begin{itemize}
    \item математически точное представление;
    \item наиболее полное описание структуры объекта.
\end{itemize}

Недостатки:
\begin{itemize}
    \item сложность;
    \item требовательность к памяти;
    \item содержит излишнюю информацию, которая не будет использована при обнаружении столкновений объектов.
\end{itemize}

\subsubsection*{Вывод}

На основе проведённого анализа различных способов представления объектов сцены, была выбрана поверхностная модель, так как она обладает всей информацией, нужной для обнаружения столкновений объектов и учёта освещения, а также является менее требовательной к памяти по сравнению с твёрдотельной моделью.

\subsection{Выбор алгоритма удаления невидимых линий и поверхностей}

\subsubsection{Алгоритм Робертса}

\subsubsection{Алгоритм обратной трассировки лучей}

\subsubsection{Алгоритм Варнока}

\subsubsection{Алгоритм, использующий z-буфер}

\subsubsection*{Вывод}

\subsection{Выбор алгоритмов обнаружения коллизий}

\subsubsection{Алгоритм обнаружения коллизий с использованием сфер}

\subsubsection{Алгоритм AABB}

\subsubsection{Алгоритм OBB}

\subsubsection{Алгоритм GJK}

\subsubsection*{Вывод}

\subsection{Выбор алгоритмов разрешения коллизий}

\subsubsection{Алгоритм EPA}

\subsubsection*{Вывод}

\subsection{Выбор модели освещения}

\subsubsection*{Вывод}
