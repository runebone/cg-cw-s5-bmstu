\thispagestyle{empty}

\begin{center}
    \fontsize{12pt}{12pt}\selectfont
    % \begin{minipage}{0.14\textwidth}
    %     \includegraphics[width=\linewidth]{img/bmstu_logo.jpg}
    % \end{minipage}
    % \hfill
    \begin{minipage}{\textwidth}\centering\bfseries
        {
            \linespread{1}\selectfont
            \vspace{0.1cm}
            {Министерство~науки~и~высшего~образования~Российской~Федерации}

            {Федеральное~государственное~бюджетное~образовательное~учреждение высшего~образования}

            {
                <<Московский~государственный~технический~университет

                имени~Н.~Э.~Баумана

                (национальный~исследовательский~университет)>>
            }

            {(МГТУ им. Н.~Э.~Баумана)}
            \vspace{0.1cm}
        }
    \end{minipage}

    \vspace{0.2cm}
    \rule{\linewidth}{2.8pt}
    \rule[3ex]{\linewidth}{1pt}

    % \begin{flushleft}
    %     {ФАКУЛЬТЕТ \uline{ИУ <<Информатика, искусственный интеллект и системы управления>> \hfill}}

    %     \vspace{0.5cm}

    %     {КАФЕДРА \uline{ИУ7 <<Программное обеспечение ЭВМ и информационные технологии>> \hfill}}
    % \end{flushleft}
\end{center}

\fontsize{11pt}{11pt}\selectfont
\begin{flushright}
    \begin{minipage}{0.4\textwidth}\raggedleft

        УТВЕРЖДАЮ \hspace{2.5cm}

        Заведующий кафедрой~\ulinetext[2cm]{(Индекс)}{ИУ7}

        % \vspace{0.3cm}

        \ulinetext{}{} \ulinetext{\hfill (И. О. Фамилия)}{И. В. Рудаков}

        % \vspace{0.3cm}

        << \ulinetext[1cm]{}{} >> \ulinetext{}{} \the\year \ г.
    \end{minipage}
\end{flushright}

\begin{center}\linespread{1}\selectfont
    \Large{\textbf{ЗАДАНИЕ}}

    \large{\textbf{на выполнение курсовой работы}}
\end{center}

\noindent по дисциплине \uline{\hspace{1cm} Компьютерная графика \hfill}

\noindent Студент, группа \uline{\hspace{1cm} Рунов Константин Алексеевич, ИУ7-54Б \hfill}

% \vspace{0.3cm}

% \noindent \ulinetext[\textwidth]{(Фамилия, имя, отчество)}{}

% \vspace{0.3cm}

\noindent \begin{tabularx}{\linewidth}{@{}X@{}}
Тема курсовой работы \uline{
    Разработка программы для моделирования столкновений объектов в виртуальном пространстве.
    \hfill
    }
\end{tabularx}

% \noindent \uline{\hfill}

% \vspace{0.3cm}

\noindent Направленность КР (учебная, исследовательская, практическая, производственная, др.)
\noindent \uline{\hfill учебная \hfill}

\noindent Источник тематики (кафедра, предприятие, НИР)
\noindent \ulinetext[8.3cm]{}{\hfill каферда \hfill}

% \vspace{0.3cm}

\noindent График выполнения работы: 25\% к \ulinetext[0.5cm]{}{4} нед., 50\% к \ulinetext[0.5cm]{}{7} нед., 75\% к \ulinetext[0.5cm]{}{11} нед., 100\% к \ulinetext[0.5cm]{}{14} нед.

% \vspace{0.3cm}

% @{} to eliminate edges of tabular
\noindent \begin{tabularx}{\linewidth}{@{}X@{}}
    \textbf{\textit{1. Техническое задание}} \uline{
Разработать программу для моделирования столкновений объектов друг с другом. Пользователь должен иметь возможность создавать объекты: куб, шар, чайник, прямоугольный параллелепипед. Пользователь должен иметь возможность изменять свойста создаваемых объектов: размер, местоположение, цвет, масса, начальная скорость, а также иметь возможность изменять положение камеры и устанавливать значение гравитации. Программа должна моделировать реалистичное движение и столкновение объектов друг с другом, учитывая освещение сцены.
        \hfill}
\end{tabularx}

% \vspace{0.3cm}

\noindent \textbf{\textit{2. Оформление курсовой работы}}

\noindent 2.1. Расчётно-пояснительная записка на \uline{20-40} листах формата А4.

% \noindent \uline{\hfill}

% \noindent \uline{\hfill}

% \noindent \begin{tabularx}{\linewidth}{@{}X@{}}
    % \uline{
\noindent Расчетно-пояснительная записка должна содержать постановку, введение, аналитическую часть, конструкторскую часть, технологическую часть, экспериментально-исследовательский раздел, заключение, список литературы, приложения.%\hfill
        % \hfill}
% \end{tabularx}

\noindent 2.2. Перечень графического материала (плакаты, схемы, чертежи и т.п.)

% \noindent \begin{tabularx}{\linewidth}{@{}X@{}}
    % \uline{
\noindent На защиту проекта должна быть представлена презентация, состоящая из 15-20 слайдов. На слайдах должны быть отражены: постановка задачи, использованные методы и алгоритмы, расчётные соотношения, структура комплекса программы, диаграмма классов, интерфейс, характеристика разработанного ПО, результаты проведённых исследований.%\hfill
        % \hfill}
% \end{tabularx}


% \vspace{0.3cm}

\noindent Дата выдачи задания << \uline{\hspace{0.5cm}} >> \uline{\hspace{2cm}} \the\year \ г.

% \vspace{0.3cm}

\noindent \textbf{Руководитель курсовой работы} \hfill \ulinetext{(Подпись, дата)}{} \ulinetext[4cm]{\hfill (И. О. Фамилия)}{\hfill А. А. Павельев}

% \vspace{0.3cm}

\thispagestyle{empty}

\noindent \textbf{Студент} \hfill \ulinetext{(Подпись, дата)}{} \ulinetext[4cm]{\hfill (И. О. Фамилия)}{\hfill К. А. Рунов}

\vfill
\fontsize{12pt}{12pt}\selectfont
